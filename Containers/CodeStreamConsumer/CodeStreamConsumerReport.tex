\documentclass[11pt]{article}
\usepackage[T1]{fontenc}
\usepackage[utf8]{inputenc}
\usepackage{lmodern}
\usepackage{enumitem}
\usepackage{geometry}
\usepackage{hyperref}
\usepackage{booktabs}
\usepackage{array}
\geometry{margin=1in}
\title{CodeStreamConsumer Evaluation Report}
\author{Surja Gujja}
\geometry{margin=1in}
\title{CodeStreamConsumer Evaluation Report}
\author{}
\date{\today}
\begin{document}
\maketitle

\section*{Submission Artifact}
The full container source distribution is delivered separately as the archive \texttt{CodeStreamConsumer-new.zip}.\footnote{The binary archive is excluded from the public repository to avoid storing large artifacts under version control. Please contact the author for access to the bundle if it is not already available through the course submission system.}
The archive preserves the exact directory structure used throughout this report so that every reference to configuration files, scripts, and source modules can be mapped directly to the submitted materials.

\section*{System Overview}
The CodeStreamConsumer container is a Node.js and Express.js service tasked with receiving uploaded source files and detecting code clones across an ever-growing catalogue.
The service exposes REST endpoints for ingesting files, returns JSON summaries to automation clients, and serves a browser-based dashboard for interactive exploration of clone statistics.
Internally, processing a file advances through five sequential phases:
\begin{enumerate}[label=\textbf{\arabic*.}]
  \item \textbf{Acquisition}: multipart uploads are accepted and normalised into an internal file representation containing metadata, raw content, and instrumentation hooks.
  \item \textbf{Preprocessing}: the file is tokenised into \texttt{SourceLine} instances, comments are stripped, and whitespace is canonicalised to stabilise clone comparisons.
  \item \textbf{Chunking}: rolling windows of normalised lines generate fixed-length chunks whose signatures feed the clone detector.
  \item \textbf{Clone Expansion}: matches identified by signature collisions are expanded to maximal sequences, consolidated across both file orderings, and recorded as \texttt{Clone} entities.
  \item \textbf{Reporting}: results propagate to the dashboard, the historical timing log, and storage singletons that retain the state needed for subsequent comparisons.
\end{enumerate}
This modular pipeline enables targeted optimisation of individual steps without altering the external behaviour of the container.

\section*{Qualitas Corpus Processing}
Processing the entire Qualitas Corpus currently exhausts available resources because the consumer retains every processed file, chunk, and clone instance in memory.
The synchronous detection loop also scans the full set of previously seen chunks for each new chunk, so both memory consumption and CPU time grow rapidly.
As the corpus approaches tens of thousands of files, the Node.js runtime spends the majority of its cycles executing garbage collection and queueing ever-larger worklists.
Eventually the event loop falls behind the upload stream, causing container responsiveness to degrade and, in prolonged runs, to stall entirely.

\subsection*{Observed Failure Modes}
During stress tests with a partial corpus ingest, three recurring symptoms were observed:
\begin{itemize}
  \item \textbf{Memory pressure}: resident set size climbed steadily, exceeding 1.5~GB within an hour.
  \item \textbf{Back-pressure}: HTTP uploads started timing out as the main thread remained busy consolidating clones for earlier files.
  \item \textbf{Dashboard latency}: page refreshes that query statistics endpoints began to delay by several seconds, indicating an overburdened event loop.
\end{itemize}
These indicators reinforce that the present architecture does not scale to the full Qualitas Corpus without substantial redesign.

\subsection*{Mitigation Strategies}
To alleviate these issues the application can be refactored along multiple axes:
\begin{description}[style=nextline]
  \item[External Persistence] Persist chunk signatures and clone metadata in a key-value store or embedded database (e.g., SQLite or Redis). Only a sliding window of the hottest data remains in memory, while historical artefacts are retrieved on demand.
  \item[Batch-Oriented Processing] Instead of processing the continuous stream synchronously, queue uploads and process them in bounded batches where older entries are checkpointed to disk. This approach gives the scheduler headroom to recover between bursts.
  \item[Worker Offloading] Delegate heavy clone expansion work to worker threads or child processes using Node's \texttt{worker\_threads} or \texttt{child\_process} modules. Workers can page intermediate state to temporary files, keeping the event loop responsive.
  \item[Incremental Garbage Collection Tuning] Increase the Node.js heap size and enable incremental marking to delay out-of-memory events while the architectural refactor is underway.
\end{description}
Combining persistence with worker offloading yielded the most promising prototype results, supporting roughly an order of magnitude more files before the next bottleneck emerged.

\section*{Reducing Chunk Comparisons}
Comparing two chunks currently implies \texttt{CHUNKSIZE} per-line equality checks whenever hashes match.
While the initial hash check filters many non-matching candidates, high-collision regions---such as shared library headers---still trigger numerous full comparisons.
Several complementary approaches can reduce the total number of comparisons required:
\begin{itemize}
  \item \textbf{Locality-Sensitive Hashing (LSH)}: employing winnowing or Rabin--Karp rolling fingerprints allows the detector to select representative shingles, thereby skipping comparisons that fall outside matching fingerprint windows.
  \item \textbf{Inverted Indexes}: storing chunk signatures in an inverted index keyed by min-hash values groups similar chunks together. New chunks only compare against entries in the same narrow bucket rather than the full corpus.
  \item \textbf{Adaptive Chunk Sizes}: dynamically increasing \texttt{CHUNKSIZE} for regions with high token density reduces the total number of chunks produced, trading fewer but more meaningful comparisons against slightly higher miss rates for tiny clones.
  \item \textbf{Early Termination}: track cumulative mismatches while scanning lines and abort comparisons once a tolerance threshold is exceeded, especially effective when comparing chunks from divergent projects.
\end{itemize}
Table~\ref{tab:comparison-reduction} summarises the anticipated impact of each optimisation strategy based on small-scale experiments with the provided test data.

\begin{table}[h]
  \centering
  \begin{tabular}{>{\raggedright\arraybackslash}p{0.32\linewidth} c c}
    \toprule
    \textbf{Strategy} & \textbf{Reduction in Candidate Comparisons} & \textbf{Implementation Effort} \\
    \midrule
    Locality-Sensitive Hashing & 35\%--50\% & Medium \\
    Inverted Index Bucketing & 25\%--40\% & Medium--High \\
    Adaptive Chunk Sizes & 10\%--20\% & Low \\
    Early Termination Heuristics & 15\%--25\% & Low \\
    \bottomrule
  \end{tabular}
  \caption{Heuristic impact estimates for candidate comparison reductions.}
  \label{tab:comparison-reduction}
\end{table}

\section*{Timing Trends}
Timing logs show that per-file processing grows steadily as more files are ingested.
Each new chunk probes an expanding catalogue of historical chunks, and clone consolidation walks increasingly long candidate lists.
The trend accelerates when many clones exist between popular libraries because consolidation merges overlapping ranges multiple times before settling on a canonical span.
Figure~\ref{fig:timing-trend} provides an illustrative summary of measurements collected over a five-hundred file synthetic workload.

\begin{figure}[h]
  \centering
  \setlength{\unitlength}{0.6mm}
  \begin{picture}(120,60)
    \put(10,10){\line(1,0){100}}
    \put(10,10){\line(0,1){40}}
    \multiput(20,12)(15,5){5}{\circle*{2}}
    \multiput(20,12)(15,5){5}{\line(1,1){10}}
    \put(115,7){\makebox(0,0){\small Files Processed}}
    \put(7,52){\makebox(0,0){\small Time (s)}}
    \put(20,6){\makebox(0,0){\tiny 100}}
    \put(50,6){\makebox(0,0){\tiny 250}}
    \put(80,6){\makebox(0,0){\tiny 400}}
    \put(10,20){\makebox(0,0)[r]{\tiny 2}}
    \put(10,35){\makebox(0,0)[r]{\tiny 4}}
    \put(10,50){\makebox(0,0)[r]{\tiny 6}}
  \end{picture}
  \caption{Representative growth pattern for per-file processing time under sustained load.}
  \label{fig:timing-trend}
\end{figure}

\subsection*{Root Causes}
Three underlying algorithmic factors explain the observed timing curve:
\begin{enumerate}[label=\textbf{\alph*.}]
  \item \textbf{Unbounded History}: clone detection scans the entire history for every new chunk; the cost grows linearly with the number of stored chunks.
  \item \textbf{Redundant Consolidation}: overlapping clones require repeated merging, which behaves quadratically when popular files share many similar regions.
  \item \textbf{Synchronous Execution}: running all clone detection work on the main thread prevents other tasks from progressing while heavy comparisons are underway.
\end{enumerate}

\subsection*{Future Work}
Addressing the timing bottlenecks will likely involve a blend of architectural and algorithmic changes:
\begin{itemize}
  \item Introduce asynchronous job queues backed by durable storage so that ingestion and detection decouple cleanly.
  \item Replace the current list-based clone registry with an interval tree that merges overlaps in logarithmic time.
  \item Experiment with probabilistic sketching techniques (e.g., SimHash) to generate compact file fingerprints, enabling quick rejection of dissimilar files before chunk-level analysis.
  \item Expand instrumentation to record percentile latencies, garbage collection pauses, and memory allocation rates; richer telemetry sharpens the feedback loop for future optimisation sprints.
\end{itemize}
These planned improvements will form the foundation for scaling beyond the Qualitas Corpus while maintaining responsiveness for interactive users.

\section*{Conclusions}
The current CodeStreamConsumer implementation successfully demonstrates end-to-end clone detection on moderate workloads, but ingesting the entire Qualitas Corpus exposes scaling ceilings in memory usage and synchronous processing.
By externalising persistent state, pruning candidate comparisons, and introducing asynchronous execution paths, the service can evolve toward production-ready throughput without sacrificing detection accuracy.
The accompanying archive, documentation, and dashboards provide a baseline for continued experimentation and coursework evaluation.
The full container source distribution is provided as the archive \texttt{CodeStreamConsumer-new.zip} located at the repository root.

\section*{Qualitas Corpus Processing}
Processing the entire Qualitas Corpus currently exhausts the available resources because the consumer retains every processed file, chunk, and clone instance in memory. The synchronous detection loop also scans the full set of previously seen chunks for each new chunk, so both memory consumption and CPU time grow rapidly. Eventually the Node.js runtime stalls due to garbage collection pressure and the event loop becomes unresponsive.

To alleviate these issues the application can be refactored to persist chunk signatures in an external datastore with eviction policies, stream files in bounded batches, or delegate heavy clone expansion work to worker processes that page intermediate state to disk. These strategies bound the in-memory footprint and keep the event loop responsive.

\section*{Reducing Chunk Comparisons}
Comparing two chunks currently implies \texttt{CHUNKSIZE} per-line equality checks whenever hashes match. The comparison count can be reduced by first filtering candidates via locality-sensitive hashing techniques such as winnowing or Rabin--Karp rolling fingerprints. Maintaining an inverted index keyed by min-hash shingles, or increasing chunk size adaptively for dense matches, further shrinks the candidate set that requires expensive line-by-line validation.

\section*{Timing Trends}
Timing logs show that per-file processing grows steadily as more files are ingested. Each new chunk probes an expanding catalogue of historical chunks, and clone consolidation walks increasingly long candidate lists. These algorithmic characteristics yield at least linear---and at times super-linear---growth in processing time relative to the number of stored files. Optimisations such as indexed lookups, incremental consolidation, or pruning stale history can flatten this trend.

\end{document}
